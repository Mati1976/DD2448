\documentclass[11pt,twoside]{article}
\usepackage[T1]{fontenc}
\usepackage[latin1]{inputenc}
\usepackage[english]{babel}
\usepackage{times}
\usepackage{amsmath}
\usepackage{amssymb}
\usepackage{amsthm}
\usepackage{url}
\usepackage{hyperref}
\usepackage{graphicx}
\usepackage{multirow}
\usepackage{tabularx}
\usepackage{fancyhdr}
\usepackage{lastpage}
\usepackage[a4paper,margin=2.5cm,hmarginratio=1:1]{geometry}

%%%%%%%%%%%%%%%%%%%%%%%%%%%%%%%%%%%%%%%%%%%%%%%%%%%%%%%%%%%%%%%%%%%%%%%%%%%
%%%%%%%%%%%%%% ENTER YOUR PERSONAL INFORMATION HERE %%%%%%%%%%%%%%%%%%%%%%%
%%%%%%%%%%%%%%%%%%%%%%%%%%%%%%%%%%%%%%%%%%%%%%%%%%%%%%%%%%%%%%%%%%%%%%%%%%%

% The info of your group members. Set Y and Z empty if you are not
% three.

% THIS FIRST ONE IS YOU.
\newcommand{\persnrX}{123456-7890}
\newcommand{\nameX}{Alan}
\newcommand{\familynameX}{Turing}
\newcommand{\emailX}{alan@turing.hi}

% THE OTHER TWO ARE PEOPLE YOU HAVE DISCUSSED INFORMALLY WITH.
\newcommand{\persnrY}{123456-7890}
\newcommand{\nameY}{Mohammad}
\newcommand{\familynameY}{Al-Khwarizmi}
\newcommand{\emailY}{mohammad@alkhwarizmi.hi}

\newcommand{\persnrZ}{123456-7890}
\newcommand{\nameZ}{Ada}
\newcommand{\familynameZ}{Lovelace}
\newcommand{\emailZ}{ada@lovelace.hi}


%%%%%%%%%%%%%%%%%%%%%%%%%%%%%%%%%%%%%%%%%%%%%%%%%%%%%%%%%%%%%%%%%%%%%%%%%%%
%%%%%%%%%%%%% DO NOT TOUCH ANYTHING BELOW THIS LINE %%%%%%%%%%%%%%%%%%%%%%%
%%%%%%%%%%%%%%%%%%%%%%%%%%%%%%%%%%%%%%%%%%%%%%%%%%%%%%%%%%%%%%%%%%%%%%%%%%%

%%%%%%%%%%%%% Environments %%%%%%%%%%%%%%%

\makeatletter
\def\th@definition{%
  \thm@notefont{\bfseries}% same as heading font
  \normalfont % body font
}
\makeatother

\theoremstyle{definition}
\newtheorem{amsproblem}{Problem}
\newtheorem{amssubproblem}{Task}[amsproblem]

\newenvironment{problem}[1][]{%
  \begin{amsproblem}[#1]
  }{%
  \end{amsproblem}
}

\newenvironment{subproblem}[1][]{%
  \begin{amssubproblem}[#1]
  }{%
  \end{amssubproblem}
}

\newcommand{\homeworknr}{I}
\newcommand{\homework}{Homework}
\newcommand{\coursenumber}{DD2448}
\newcommand{\coursename}{\coursenumber~Foundations of cryptography}
\newcommand{\coursenick}{krypto20}

\lhead[\familynameX~\familynameY~\familynameZ]{\coursename}
\chead{}
\rhead[\coursename]{\familynameX~\familynameY~\familynameZ}
\lfoot[\thepage~(\pageref{LastPage})]{}
\cfoot{}
\rfoot[]{\thepage~(\pageref{LastPage})}

\fancypagestyle{firststyle}
{
   \fancyhf{}
   \fancyfoot[R]{\thepage~(\pageref{LastPage})}
}

\renewcommand{\headrulewidth}{0pt}

\newcommand{\TP}[1]{#1T}
\newcommand{\IP}[1]{#1I}


%%%%%%%%%%%%%%%%%%%%%%%%%%%%%%%%%%%%%%%%%%%%%%%%%%%%%%%%%%%%%%%%%%%%%%%%%%%
%%% HERE YOU CAN ADD YOUR OWN MACROS AND ENVIRONMENTS IN THE PREAMBLE %%%%%
%%%%%%%%%%%%%%%%%%%%%%%%%%%%%%%%%%%%%%%%%%%%%%%%%%%%%%%%%%%%%%%%%%%%%%%%%%%

% Add your macros here.

\newcommand{\TPOINTS}[1]{(#1T)}
\newcommand{\IPOINTS}[1]{(#1I)}

\begin{document}

%%%%%%%%%%%%%%%%%%%%%%%%%%%%%%%%%%%%%%%%%%%%%%%%%%%%%%%%%%%%%%%%%%%%%%%%%%%
%%%%%%%%%%%% THE FOLLOWING GENERATES THE HEADER %%%%%%%%%%%%%%%%%%%%%%%%%%%
%%%%%%%%%%%% DO NOT TOUCH THIS %%%%%%%%%%%%%%%%%%%%%%%%%%%%%%%%%%%%%%%%%%%%
%%%%%%%%%%%%%%%%%%%%%%%%%%%%%%%%%%%%%%%%%%%%%%%%%%%%%%%%%%%%%%%%%%%%%%%%%%%

\thispagestyle{firststyle}

\noindent
\hspace{0.3cm}{\huge\textbf{\coursename}}

\noindent
\rule{\textwidth}{1pt}

\noindent
\begin{tabularx}{\textwidth}{X|lll}
  & \textbf{Persnr} & \textbf{Name} & \textbf{Email} \\
\cline{2-4}
&\\[-0.3cm]
  \multirow{2}{*}{\textbf{\huge\homework}} & {\small\textbf{\persnrX}} & {\small\textbf{\nameX}} & {\small\textbf{\emailX}} \\
  & & {\small\textbf{\familynameX}} & \\
\cline{2-4}
  \multirow{2}{*}{\textbf{\huge\coursenick}} & {\small\persnrY} & {\small\nameY} & {\small\emailY} \\
  & & {\small\familynameY} & \\
  & {\small\persnrZ} & {\small\nameZ} & {\small\emailZ} \\
  & & {\small\familynameZ} & \\
&\\
[-0.2cm]
\end{tabularx}

\vspace{0.2cm}
\noindent
\rule{\textwidth}{1pt}

\vspace{0.5cm}

\pagestyle{fancy}

%%%%%%%%%%%%%%%%%%%%%%%%%%%%%%%%%%%%%%%%%%%%%%%%%%%%%%%%%%%%%%%%%%%%%%%%%%%
%%%%%%%%%%%%%%%%%%%%% YOUR SOLUTIONS START HERE %%%%%%%%%%%%%%%%%%%%%%%%%%%
%%%%%%%%%%%%%%%%%%%%%%%%%%%%%%%%%%%%%%%%%%%%%%%%%%%%%%%%%%%%%%%%%%%%%%%%%%%
%%                                                                       %%
%%  Do NOT remove any problem-, or subproblem environments, or nominal   %%
%%  ponts below. If you can not solve a problem, then you MUST simply    %%
%%  leave the "NOT SOLVED" string intact. This ensures that the          %%
%%  numbering is correct and it simplifies grading, leaving more time    %%
%%  to prepare lectures and help students.                               %%
%%                                                                       %%
%%%%%%%%%%%%%%%%%%%%%%%%%%%%%%%%%%%%%%%%%%%%%%%%%%%%%%%%%%%%%%%%%%%%%%%%%%%

\begin{problem}
  \begin{subproblem}[\TP{3}]
    NOT SOLVED % We leave this place holder here for improved readability.
  \end{subproblem}
  \begin{subproblem}[\TP{3}]
    NOT SOLVED % We leave this place holder here for improved readability.
  \end{subproblem}
  \begin{subproblem}[\TP{4}]
    NOT SOLVED % We leave this place holder here for improved readability.
  \end{subproblem}
\end{problem}

\noindent
\hrulefill

\begin{problem}
  \begin{subproblem}[\TP{1}]
    NOT SOLVED % We leave this place holder here for improved readability.
  \end{subproblem}
  \begin{subproblem}[\TP{2}]
    NOT SOLVED % We leave this place holder here for improved readability.
  \end{subproblem}
  \begin{subproblem}[\TP{1}]
    NOT SOLVED % We leave this place holder here for improved readability.
  \end{subproblem}
  \begin{subproblem}[\TP{1}]
    NOT SOLVED % We leave this place holder here for improved readability.
  \end{subproblem}
\end{problem}

\noindent
\hrulefill

\begin{problem}[\TP{4}]
  NOT SOLVED % We leave this place holder here for improved readability.
\end{problem}

\noindent
\hrulefill

\begin{problem}
  \begin{subproblem}[\TP{2}]
    NOT SOLVED % We leave this place holder here for improved readability.
  \end{subproblem}
  \begin{subproblem}[\TP{1}]
    NOT SOLVED % We leave this place holder here for improved readability.
  \end{subproblem}
  \begin{subproblem}[\TP{2}]
    NOT SOLVED % We leave this place holder here for improved readability.
  \end{subproblem}
  \begin{subproblem}[\TP{1}]
    NOT SOLVED % We leave this place holder here for improved readability.
  \end{subproblem}
\end{problem}

\noindent
\hrulefill

\begin{problem}
  \begin{subproblem}[\TP{5}]
    NOT SOLVED % We leave this place holder here for improved readability.
  \end{subproblem}
  \begin{subproblem}[\TP{4}]
    NOT SOLVED % We leave this place holder here for improved readability.
  \end{subproblem}
  \begin{subproblem}[\TP{4}]
    NOT SOLVED % We leave this place holder here for improved readability.
  \end{subproblem}
\end{problem}

\noindent
\hrulefill

\begin{problem}
  \begin{subproblem}[\TP{2}]
    NOT SOLVED % We leave this place holder here for improved readability.
  \end{subproblem}
  \begin{subproblem}[\TP{2}]
    NOT SOLVED % We leave this place holder here for improved readability.
  \end{subproblem}
  \begin{subproblem}[\TP{2}]
    NOT SOLVED % We leave this place holder here for improved readability.
  \end{subproblem}
  \begin{subproblem}[\TP{2}]
    NOT SOLVED % We leave this place holder here for improved readability.
  \end{subproblem}  
\end{problem}

\end{document}

%%% Local Variables:
%%% mode: latex
%%% TeX-master: t
%%% End:
