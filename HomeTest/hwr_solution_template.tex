\documentclass[11pt,twoside]{article}
\usepackage[T1]{fontenc}
\usepackage[latin1]{inputenc}
\usepackage[english]{babel}
\usepackage{times}
\usepackage{amsmath}
\usepackage{amssymb}
\usepackage{amsthm}
\usepackage{url}
\usepackage{hyperref}
\usepackage{graphicx}
\usepackage{multirow}
\usepackage{tabularx}
\usepackage{fancyhdr}
\usepackage{lastpage}
\usepackage[a4paper,margin=2.5cm,hmarginratio=1:1]{geometry}
\usepackage{calligra}

%%%%%%%%%%%%%%%%%%%%%%%%%%%%%%%%%%%%%%%%%%%%%%%%%%%%%%%%%%%%%%%%%%%%%%%%%%%
%%%%%%%%%%%%%% ENTER YOUR PERSONAL INFORMATION HERE %%%%%%%%%%%%%%%%%%%%%%%
%%%%%%%%%%%%%%%%%%%%%%%%%%%%%%%%%%%%%%%%%%%%%%%%%%%%%%%%%%%%%%%%%%%%%%%%%%%

% The info of your group members. Set Y and Z empty if you are not
% three.

% THIS FIRST ONE IS YOU.
\newcommand{\persnrX}{130576-7319}
\newcommand{\nameX}{Mati}
\newcommand{\familynameX}{Rachamim}
\newcommand{\emailX}{Rachamim@kth.se}

% THE OTHER TWO ARE PEOPLE YOU HAVE DISCUSSED INFORMALLY WITH.
\newcommand{\persnrY}{}
\newcommand{\nameY}{}
\newcommand{\familynameY}{}
\newcommand{\emailY}{}

\newcommand{\persnrZ}{}
\newcommand{\nameZ}{}
\newcommand{\familynameZ}{}
\newcommand{\emailZ}{}


%%%%%%%%%%%%%%%%%%%%%%%%%%%%%%%%%%%%%%%%%%%%%%%%%%%%%%%%%%%%%%%%%%%%%%%%%%%
%%%%%%%%%%%%% DO NOT TOUCH ANYTHING BELOW THIS LINE %%%%%%%%%%%%%%%%%%%%%%%
%%%%%%%%%%%%%%%%%%%%%%%%%%%%%%%%%%%%%%%%%%%%%%%%%%%%%%%%%%%%%%%%%%%%%%%%%%%

%%%%%%%%%%%%% Environments %%%%%%%%%%%%%%%

\makeatletter
\def\th@definition{%
  \thm@notefont{\bfseries}% same as heading font
  \normalfont % body font
}
\makeatother

\theoremstyle{definition}
\newtheorem{amsproblem}{Problem}
\newtheorem{amssubproblem}{Task}[amsproblem]

\newenvironment{problem}[1][]{%
  \begin{amsproblem}[#1]
  }{%
  \end{amsproblem}
}

\newenvironment{subproblem}[1][]{%
  \begin{amssubproblem}[#1]
  }{%
  \end{amssubproblem}
}

\newcommand{\homeworknr}{I}
\newcommand{\homework}{Homework}
\newcommand{\coursenumber}{DD2448}
\newcommand{\coursename}{\coursenumber~Foundations of cryptography}
\newcommand{\coursenick}{krypto20}

\lhead[\familynameX~\familynameY~\familynameZ]{\coursename}
\chead{}
\rhead[\coursename]{\familynameX~\familynameY~\familynameZ}
\lfoot[\thepage~(\pageref{LastPage})]{}
\cfoot{}
\rfoot[]{\thepage~(\pageref{LastPage})}

\fancypagestyle{firststyle}
{
   \fancyhf{}
   \fancyfoot[R]{\thepage~(\pageref{LastPage})}
}

\renewcommand{\headrulewidth}{0pt}

\newcommand{\TP}[1]{#1T}
\newcommand{\IP}[1]{#1I}


%%%%%%%%%%%%%%%%%%%%%%%%%%%%%%%%%%%%%%%%%%%%%%%%%%%%%%%%%%%%%%%%%%%%%%%%%%%
%%% HERE YOU CAN ADD YOUR OWN MACROS AND ENVIRONMENTS IN THE PREAMBLE %%%%%
%%%%%%%%%%%%%%%%%%%%%%%%%%%%%%%%%%%%%%%%%%%%%%%%%%%%%%%%%%%%%%%%%%%%%%%%%%%

% Add your macros here.

\newcommand{\TPOINTS}[1]{(#1T)}
\newcommand{\IPOINTS}[1]{(#1I)}

\begin{document}

%%%%%%%%%%%%%%%%%%%%%%%%%%%%%%%%%%%%%%%%%%%%%%%%%%%%%%%%%%%%%%%%%%%%%%%%%%%
%%%%%%%%%%%% THE FOLLOWING GENERATES THE HEADER %%%%%%%%%%%%%%%%%%%%%%%%%%%
%%%%%%%%%%%% DO NOT TOUCH THIS %%%%%%%%%%%%%%%%%%%%%%%%%%%%%%%%%%%%%%%%%%%%
%%%%%%%%%%%%%%%%%%%%%%%%%%%%%%%%%%%%%%%%%%%%%%%%%%%%%%%%%%%%%%%%%%%%%%%%%%%

\thispagestyle{firststyle}

\noindent
\hspace{0.3cm}{\huge\textbf{\coursename}}

\noindent
\rule{\textwidth}{1pt}

\noindent
\begin{tabularx}{\textwidth}{X|lll}
  & \textbf{Persnr} & \textbf{Name} & \textbf{Email} \\
\cline{2-4}
&\\[-0.3cm]
  \multirow{2}{*}{\textbf{\huge\homework}} & {\small\textbf{\persnrX}} & {\small\textbf{\nameX}} & {\small\textbf{\emailX}} \\
  & & {\small\textbf{\familynameX}} & \\
\cline{2-4}
  \multirow{2}{*}{\textbf{\huge Revisited}} & {\small\persnrY} & {\small\nameY} & {\small\emailY} \\
  & & {\small\familynameY} & \\
  \textbf{\huge\coursenick} & {\small\persnrZ} & {\small\nameZ} & {\small\emailZ} \\
  & & {\small\familynameZ} & \\
&\\
[-0.2cm]
\end{tabularx}

\vspace{0.2cm}
\noindent
\rule{\textwidth}{1pt}

\vspace{0.5cm}

\pagestyle{fancy}

%%%%%%%%%%%%%%%%%%%%%%%%%%%%%%%%%%%%%%%%%%%%%%%%%%%%%%%%%%%%%%%%%%%%%%%%%%%
%%%%%%%%%%%%%%%%%%%%% YOUR SOLUTIONS START HERE %%%%%%%%%%%%%%%%%%%%%%%%%%%
%%%%%%%%%%%%%%%%%%%%%%%%%%%%%%%%%%%%%%%%%%%%%%%%%%%%%%%%%%%%%%%%%%%%%%%%%%%
%%                                                                       %%
%%  Do NOT remove any problem-, or subproblem environments, or nominal   %%
%%  points below. If you can not solve a problem, then you MUST simply   %%
%%  leave the "NOT SOLVED" string intact. This ensures that the          %%
%%  numbering is correct and it simplifies grading, leaving more time    %%
%%  to prepare lectures and help students.                               %%
%%                                                                       %%
%%%%%%%%%%%%%%%%%%%%%%%%%%%%%%%%%%%%%%%%%%%%%%%%%%%%%%%%%%%%%%%%%%%%%%%%%%%

\begin{problem}
\begin{subproblem}[\TP{1}]
    Let $p_{0},...p_{l}\in [0,1]$. We can define the next sum:\newline $S_{l}=\left|p_{1}-p_{0}\right|+\left|p_{2}-p_{1}\right|+\left|p_{3}-p_{2}\right|+...+\left|p_{l-1}-p_{l-2}\right|+\left|p_{l}-p_{l-1}\right|$ \newline
    We know that this sum must follow the triangle inequality :\newline
    $S_{l}\geq \left|p_{1}-p_{0}+p_{2}-p_{1}+p_{3}-p_{2}+...+p_{l-1}-p_{l-2}+p_{l}-p_{l-1}\right|=\left| p_{l}-p_{0}\right|$.\newline
    According to the definition we also know that $\left| p_{l}-p_{0}\right|\geq \triangle$.\newline
    If we take the assumption that all the terms that build up the sum $S_{l}$ are equal to each other and to  quantity which we will define to be $x$ we can say that $S_{l}=l\times x\geq \triangle$ and therefore we can say that $x\geq\triangle/l$ and of course that x represents each of the terms in the sum and we have proved what is needed.\newline 
    
    If the most common case, in which not all the components that build up $S_{l}$ are equal to each other, then some of them, at least one, must be bigger than $x=S_{l}/l$ (which we have defined before as such.) and hence there is at least one of the components $\left|p_{i}-p_{i-1}\right|\geq\triangle/l$ .

    % We leave this place holder here for improved readability.
  \end{subproblem}
  \begin{subproblem}[\TP{1}]
    Let \textit {$Z_{n}^{(i)}=(X_{n,1},...,X_{n,i},Y_{n,i+1},...,Y_{n,l(n)})$}, So therefore : \newline
    \textit {$Z_{n}^{(0)}=(Y_{n,1},...,Y_{n,i+1},...,Y_{n,l(n)})$}, and 
    \textit {$Z_{n}^{(l(n))}=(X_{n,1},...,X_{n,i+1},...,X_{n,l(n)})$}.\newline
    The above terms were calculated based on the case that when $i=0$ the coefficients of \textit{$X_{n,i}$} will be \textit{$X_{n,0}$} and hence $i=0<1$ so none of the \textit{$X_{n,i}$} terms will be included in \textit{$Z_{n}^{(0)}$}and the \textit{$Y_{n,i}$} terms will be starting from \textit{$Y_{n,0}$} and finishing at the last term \textit{$Y_{n,l(n)}$}. The same argument can be written for the case of \textit{$Z_{n}^{(l(n))}$} with the exchange of the rolls between \textit{$X_{n,i}$} and \textit{$Y_{n,i}$}.\newline 
    We know that \textit{$D:\{0,1\}^*\rightarrow \{0,1\}$} is an algorithm that samples and outputs a single bit and that \textit{$X_{n}$} and \textit{$Y_{n}$} are independently and identically distributed, and efficiently sampleable and hence we can deduvt that :\newline 
    Pr\textit{$[D(Z_{n}^{(0)})=1]$} = Pr\textit{$[D(Y_{n})=1]$} and also : \newline 
    Pr\textit{$[D(Z_{n}^{(l(n))})=1]$} = Pr\textit{$[D(X_{n})=1]$} .\newline 
    
    
     % We leave this place holder here for improved readability.
  \end{subproblem}
  \begin{subproblem}[\TP{8}]
    NOT SOLVED % We leave this place holder here for improved readability.
  \end{subproblem}
\end{problem}

\noindent
\hrulefill

\begin{problem}
  \begin{subproblem}[\TP{2}]
    We know that by definition a function is negligible if it  smaller than the inverse of any polynomial function for a large enough n. If we multiply such a function by another polynom we will create a new polynom and by a definition the negligible function should be smaller than any such function ( possibly for a different n ) and hence the new function is also negligible .  % We leave this place holder here for improved readability.
  \end{subproblem}
  \begin{subproblem}[\TP{2}]
    If the function is not negligible then we know that  it is not smaller than any polynom we can present to it while addressing n goes to infinity . if after doing so we se multiply it by another polynom we create a new polynom .% We leave this place holder here for improved readability.
  \end{subproblem}
\end{problem}

\noindent
\hrulefill

\begin{problem}
  \begin{subproblem}[\TP{4}]
    NOT SOLVED % We leave this place holder here for improved readability.
  \end{subproblem}
  \begin{subproblem}[\TP{4}]
    NOT SOLVED % We leave this place holder here for improved readability.
  \end{subproblem}
\end{problem}

\noindent
\hrulefill

\end{document}

%%% Local Variables:
%%% mode: latex
%%% TeX-master: t
%%% End:
